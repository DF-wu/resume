%-------------------------------------------------------------------------------
%	SECTION TITLE
%-------------------------------------------------------------------------------
\cvsection{Project \& Publish}


%-------------------------------------------------------------------------------
%	CONTENT
%-------------------------------------------------------------------------------
\begin{cventries}

%---------------------------------------------------------
  \cventry
    {Presenter for <Testing for Event-Driven Microservices Based on Consumer-Driven Contracts and State Models>. IEEE xplore published} % Role
    {29th Asia-Pacific Software Engineering Conference, APSEC 2022} % Event
    {Virtual, Japan} % Location
    {Dec. 2022} % Date(s)
    {
      \begin{cvitems} % Description(s)
        \item {This is the final result of CCTS. CCTS is a testing service bewteen intergrated test and unit test in testing pyramid.}
        \item {CCTS combine logs and State model to determine the target system behavior. In order to remain the flexibility of data structure, NoSQL is the better database type. We use MongoDB as the data persistence solution because the plenty resources and widely used.}
        \item {We validate this service to measure if it works. The result shows the CCTS does detects the potential executing failures and some common errors while testing.}
      \end{cvitems}
    }

%---------------------------------------------------------
  \cventry
    {Presenter for <Composite Contract Testing Mechanism for Event-Driven Microservices>.} % Role
    {Best Paper, 18th Taiwan Conference on Software Engineering, TCSE 2022} % Event
    {Taipei} % Location
    {Jun. 2022} % Date(s)
    {
      \begin{cvitems} % Description(s)
        % \item {Introduced a study that test a Event-driven microservice system with contract testing}
        \item {Event-driven microservice is a new software architecture but the testing is hard due to asynchronization. We proposed a new test process to test an EDA based system.}
        \item {We use state model to describe how a service interact with other services and proposed a method to effectively test an Event-driven microservice system. It reduces the efforts for testers by detecting potential failure automatically.}
      \end{cvitems}
    }

%---------------------------------------------------------
  \cventry
    {Presenter for <Design for Personal Data Authorization System and two-way payment platform>.} % Role
    {2021 National Computer Symposium, NCS 2021} % Event
    {TaiChung, Taiwan} % Location
    {Dec. 2021} % Date(s)
    {
      \begin{cvitems} % Description(s)
        \item {The PDVPS provides a thrid-party finance service for PDAS. It's designed for end-users to sell their data and withdraw from platform. This system is designed with event-driven microservice to improve reliability and throughput.}
        \item {PDVPS is a distributed point and finance system, that's how to ensure data integrity is a critical issue. We deploy MongoDB Replica Set in k8s platform as data persistence. Also, we use mongoDB Aggregation to query necessary data.}
      \end{cvitems}
    }

%---------------------------------------------------------
  \cventry
    {Presenter for <PDAS: A Digital-Signature-Based Authorization Platform for Digital Personal Data>. IEEE xplore published.} % Role
    {2020 International Computer Symposium, ICS 2020} % Event
    {Tainan, Taiwan} % Location
    {Dec. 2020} % Date(s)
    {
      \begin{cvitems} % Description(s)
        % \item {Introduced a study that how to ensure a digital authorization is legal and reliable for a digital contract with personal data.}
        \item {For the "Smart Disclosure" and "Green Button" project which is from US-government. Make users authorize their data safely and conveniently with undeniable legal contract.}
        \item {We Proposed the PDAS for digital personal data authorization, a service support Citizen Digital Certificat (IC-card) signature encrypt/decrypt with legal contracts and public block-chain for validation.}
      \end{cvitems}
    }

%---------------------------------------------------------
\end{cventries}
